\documentclass[11pt]{amsart}

\usepackage{physics}
\usepackage{amsmath}

\renewcommand{\thesubsection}{\thesection.\alph{subsection}}

\title{Relativistic Kinematics \\
	\hrulefill \small{ FYS3120: Problem Set 8 } \hrulefill}

\author[Winther-Larsen]{Sebastian G. Winther-Larsen}

\date{\today}

\begin{document}

\maketitle

\section{Mirror Mirror on the (Moving) Wall}

A monocromatic light source is at rest in the laboratory and sends photons with frequency $\nu_0$ towards a mirror which has its reflective surface perpendicular to the beam direction. The mirror moves away from the light source with velocity $v$. The transformation formula for four-momentum is given by $p^{\mu} = (E/c, \vb{p})$ and the Planck relation is $E=h\nu$.

\subsection{Light Frequency in Rest Frame of Mirror}
The relativistic energy of a moving particle is 
\begin{equation}
E = \sqrt{p^2c^2 + m^2c^4}.
\end{equation}
Because a photon is without mass, the energy of a photon according to the formula above is
\begin{equation}
E = pc,
\end{equation}
which can be inserted into Planck relation yielding
\begin{equation}
p = \frac{h \nu_0}{c}.
\end{equation}
This provides an expression for the four vector
\begin{equation}
p^\mu = \left(\frac{E}{c}, \vb{p} \right) = \left(\frac{h\nu_0}{c} \right) = (p,p,0,0).
\end{equation} 
To get from emitted frequency $\nu_0$ in lab reference frame $S$, to frequency $\nu$ in mirror reference frame $S'$ one needs to take the Lorentz transform 
\begin{equation}
p'^\mu = L^\mu_{\ \rho} p^\rho,
\end{equation}
because the mirror reference frame is just a boost along the $x$-axis, relative to the lab reference frame.
\begin{equation}
\begin{pmatrix}
p'^0 \\ p'^1 \\ p'^2 \\ p'^3
\end{pmatrix}
=
\begin{pmatrix}
\gamma & -\beta\gamma & 0 & 0 \\
-\beta\gamma & \gamma & 0 & 0 \\
0 & 0 & 1 & 0 \\
0 & 0 & 0 & 1
\end{pmatrix}
\begin{pmatrix}
p^0 \\ p^1 \\ p^2 \\ p^3
\end{pmatrix}
=
\gamma(1-\beta)
\begin{pmatrix}
p \\ p \\ 0 \\ 0
\end{pmatrix},
\end{equation}
so
\begin{equation}
p' = \gamma(1-\beta)p.
\end{equation}
The de Broglie relations gives
\begin{equation}
p = \frac{h}{\lambda} = \frac{h\nu}{c},
\end{equation}
so the frequency of the emitted and reflected light in the rest frame of the mirror must be
\begin{equation}
\nu' = \gamma(1-\beta)\nu.
\end{equation}
The frequency of the emitted and reflected light must necessarily be the same, due to conservation of momentum.

\subsection{Frequency of Reflected Light in Lab System}

Denoting frequency of reflected light as $\nu_R$ and frequency of emitted light as $\nu_0$, we already have that
\begin{equation}
\label{eq:mirrorRF}
\nu'_R = \gamma(1-\beta)\nu_0,
\end{equation}
in the mirror rest frame. Similarly, the frequency of reflected light in laboratory rest frame is
\begin{equation}
\label{eq:labRF}
\nu_R = \gamma(1-\beta)\nu'_R.
\end{equation}
Inserting \ref{eq:mirrorRF} into \ref{eq:labRF} yields
\begin{equation}
\nu_R = \gamma^2(1-\beta)^2\nu_0 = \frac{(1-\beta)^2}{1-\beta^2}\nu_0 = \frac{(1-\beta)^2}{(1+\beta)(1-\beta)}\nu_0 = \frac{1-\beta}{1+\beta}\nu_0
\end{equation}

\end{document}