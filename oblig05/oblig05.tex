\documentclass[11pt]{amsart}

\usepackage[utf8]{inputenc}
\usepackage{amsmath}
\usepackage{physics}

\renewcommand{\thesubsection}{\thesection.\alph{subsection}}

\title[Brachistochrone Problem]{Hamiltonian Dynamics of the Brachistochrone Problem\\
	\hrulefill \small{ Problem Sheet 5: FYS3120 } \hrulefill}

\author[Winther-Larsen]{Sebastian G. Winther-Larsen}

\date{\today}

\begin{document}

\maketitle

\section{Coriolis and Centrifugal forces}
A particle with mass $m$ moves freely on a horizontal plane. There are no constraints, but in the following we will consider the free motion described in a rotating reference frame. We refer to the Cartesian coordinates of a fixed frame as $(x,y)$ and the coordinates of the rotating frame as $(\xi, \eta)$. They are related by the standard expressions
\begin{align}
x &= \xi\cos\omega t - \eta\sin\omega t, \\
y &= \xi\sin\omega t + \eta\cos\omega t,
\end{align}
where $\omega$ is the angular velocity of the rotation.

\subsection{Lagrangian}


\end{document}