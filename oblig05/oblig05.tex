\documentclass[11pt]{amsart}

\usepackage[utf8]{inputenc}
\usepackage{amsmath}
\usepackage{physics}
\usepackage{csquotes}
\usepackage{graphicx}
\usepackage{hyperref}

\renewcommand{\thesubsection}{\thesection.\alph{subsection}}

\title[Brachistochrone Problem]{Hamiltonian Dynamics of the Brachistochrone Problem\\
	\hrulefill \small{ Problem Sheet 5: FYS3120 } \hrulefill}

\author[Winther-Larsen]{Sebastian G. Winther-Larsen}

\date{\today}

\begin{document}

\maketitle

\section{Coriolis and Centrifugal Forces}
A particle with mass $m$ moves freely on a horizontal plane. There are no constraints, but in the following we will consider the free motion described in a rotating reference frame. We refer to the Cartesian coordinates of a fixed frame as $(x,y)$ and the coordinates of the rotating frame as $(\xi, \eta)$. They are related by the standard expressions
\begin{align}
x &= \xi\cos\omega t - \eta\sin\omega t, \\
y &= \xi\sin\omega t + \eta\cos\omega t,
\end{align}
where $\omega$ is the angular velocity of the rotation.

\subsection{Lagrangian}
First we need $\dot{x}$ and $\dot{y}$;
\begin{align*}
\dot{x} &= \dot{\xi}\cos\omega t - \omega\xi\sin\omega t - \dot{\eta}\sin\omega t - \omega\eta\cos\omega t \\ 
		&= (\dot{\xi} - \omega\eta)\cos\omega t - (\omega\xi + \dot{\eta})\sin\omega t, \\
\dot{y} &= \dot{\xi}\sin\omega t + \omega\xi\cos\omega t + \dot{\eta}\cos\omega t - \omega\eta\sin\omega t \\
 		&= (\dot{\xi} - \omega\eta)\sin\omega t + (\omega\xi + \dot{\eta})\cos\omega t,
\end{align*}
we also need their squares
\begin{align*}
\dot{x}^2 &= (\dot{\xi}-\omega\eta)^2\cos^2\omega t - 2(\dot{\xi}- \omega\eta)(\omega\xi + \dot{\eta})\cos\omega t\sin\omega t + (\omega\xi + \dot{\eta})^2\sin^2\omega t, \\
\dot{y}^2 &= (\dot{\xi}-\omega\eta)^2\sin^2\omega t + 2(\dot{\xi} - \omega\eta)(\omega\xi + \dot{\eta})\sin\omega t \cos\omega t + (\omega\xi + \dot{\eta})^2\cos^2\omega t,
\end{align*}
the sum of the squares is
\begin{equation*}
\dot{x}^2+\dot{y}^2 = (\dot{\xi}-\omega\eta)^2 + (\omega\xi + \dot{\eta})^2 
=dot{\xi}^2 - 2\dot{\xi}\omega\eta + \omega^2\eta^2 + \omega^2\xi^2 + 2\omega\xi\dot{\eta} + \dot{\eta}^2,
\end{equation*}
which can now be used to find the Lagrangian
\begin{equation}
\label{eq:lagrangian1}
L = T = \frac{1}{2}m[\dot{\xi}^2 + \dot{\eta}^2 + \omega^2(\xi^2+\eta^2) + 2\omega(\xi\dot{\eta}- \dot{\xi}\eta)].
\end{equation}
As there is no gravity there is no potential, V.

\subsection{Equations of Motion} The general Lagrange equation is given by
\begin{equation}
\label{eq:lagrange}
\frac{d}{dt}\left(\frac{\partial L}{\partial\dot{q}_j} \right) - \frac{\partial L}{\partial q_j} = 0,
\end{equation}
and can be found for every generalised coordinate.
\subsubsection{Lagrange's equation for $\xi$}
Start by finding all parts of equation \ref{eq:lagrange}
\begin{align*}
\frac{\partial L}{\partial \xi} &= m\omega^2\xi + m\omega\dot{\eta}, \\
\frac{\partial L}{\partial \dot{\xi}} &= m\dot{\xi}-m\omega\eta, \\
\frac{d}{dt}\left(\frac{\partial L}{\partial\dot{\xi}} \right) &= m\ddot{\xi}-m\omega\dot{\eta}.
\end{align*}
Combining all these gives Lagrange's equation for $\xi$
\begin{equation}
\label{eq:lagrangexi}
m\ddot{\xi} = m\omega^2\xi + 2m\omega\dot{\eta}.
\end{equation}

\subsubsection{Lagrange's equation for $\eta$}
I am going to make an implicit symmetry argument here\footnote{Did you notice it?} and simply write down the Lagrange equation for $\eta$
\begin{equation}
\label{eq:lagrangeeta}
m\ddot{\eta} = m\omega^2\eta - 2m\omega\dot{\xi}
\end{equation} 

\subsubsection{Analysis}
Notice that I did not write down the Lagrange equations for $\xi$ and $\eta$, given by equations \ref{eq:lagrangexi} and \ref{eq:lagrangeeta} respectively, in the conventional way dictated by equation \ref{eq:lagrange}. The reason for this is that the Lagrange equations found in this problem are incredibly simlilar to Newton's second law for rotational coordinates,
\begin{equation}
\label{eq:N2Lrot}
\vb{F} = m \ddot{\vb{\rho}} = \vb{F}_{\text{imp}} + \vb{F}_{\text{centrifugal}} + \vb{F}_{\text{Coriolis}} + \vb{F}_{\text{Euler}},
\end{equation}
where $\vb{F}_{\text{imp}}$ are the forces impressed on the system ($=0$ here), $\vb{F}_{\text{centrifugal}}= -m\vb{\omega}\times(\vb{\omega}\times\vb{\rho})$ is the centrifugal force, $\vb{F}_{\text{Coriolis}} = -2m\vb{\omega}\cross\dot{\vb{\rho}}$ is the Coriolis force and $\vb{F}_{\text{Euler}}$ is the Euler force, felt in reaction to any acceleration (also $=0$ here). In all forces $\rho$ is the position vector in the rotating frame. Equation \ref{eq:N2Lrot} becomes
\begin{equation}
\label{eq:N2Lrot2}
m \ddot{\vb{\rho}} = -m\vb{\omega}\times(\vb{\omega}\times\vb{\rho}) - 2m\vb{\omega}\cross\dot{\vb{\rho}},
\end{equation}
which is very similar to the equations of motion in \ref{eq:lagrangexi} and \ref{eq:lagrangeeta}.

\section{The Brachistochrone Challenge}
The problem as posed to Isaac Newton, amongst others, in 1696 can be formulated in the following way,

\begin{displayquote}
Given two points A and B in a vertical plane, what is the curve
traced out by a body acted on only by gravity, which starts at
A and reaches B in the shortest time.
\end{displayquote}

It is said that Newton had a solution already the following day. Herein the challenge is the following, can the problem be solved using the correspondence between the variational problem and the Lagrange equation? The body $P$ is treated as a point particle of mass $m$ and the path is represented by a function $y(x)$ with $x$ as the horizontal axis and $y$ as the vertical axis. The boundary conditions, which fix the positions of point $A$ and $B$, are specified as $y(x_A) = y_A$, $y(x_b) = y_b$. A simplifications is to assume that $x_A = y_A$.

\begin{figure}
\centering
	\includegraphics[width=0.9\textwidth]{pearl_on_string.png}
	\caption{Illustration of the Brachistochrone problem}
	\label{fig:brach}
\end{figure}

\subsection{The Period of the Motion}
The total time the body spends moving along the curve is given by a simple integral of infinitesimal time steps
\begin{equation}
T = \int_{t_a}^{t_b}dt.
\end{equation}
Using the velocity and displacement relations $vdt = ds$ gives
\begin{equation}
\label{eq:brachtime2}
T = \int_{A}^{B}\frac{1}{v}ds.
\end{equation}
One such infinitesimal displacement can be decomposed in a Pythagorean manner into $x$ and $y$ parts, $ds = \sqrt{dx^2 + dy^2}$. Inserting into \ref{eq:brachtime2} yields
\begin{equation}
\label{eq:brachtime3}
T = \int_{A}^{B}\frac{1}{v}\sqrt{dx^2 + dy^2} = \int_{x_a}^{x_b}\frac{1}{v}\sqrt{1 + \left(\frac{dy}{dx} \right)^2}dx.
\end{equation}
By assuming conservations of energy $\frac{1}{2}mv^2 + mgy = 0 \to v = \sqrt{-2gy}$ and setting $y'=\frac{dy}{dx}$ equation \ref{eq:brachtime3} becomes
\begin{equation}
T = \int_{x_a}^{x_b} \sqrt{\frac{1+y'^2}{-2gy}}dx.
\end{equation}
If one were to consider this an action integral, then the integrand must therefore be the Lagrangian for the system with $y$ and $y'$ as generalised coordinates
\begin{equation}
\label{eq:lagrangian}
L(y,y') = \sqrt{\frac{1+y'^2}{-2gy}}
\end{equation}

\subsection{Hamiltonian and Differential Equation for the Problem}
Notice that the Lagrangian (equation \ref{eq:lagrangian}) does not depend explicitly on $x$, which has taken the role as  $t$ in this problem. It follows from the Lagrangian-Hamiltonian-relationship
\begin{equation}
\frac{dH}{dx} = -\frac{dL}{dx} = 0
\end{equation}
that the Hamiltonian, $H$, is a constant of motion. Then one can use the conjugate momentum as is given by $p = \frac{\partial L}{\partial y'}$ and write the Hamiltonian as $H = py' - L$. The conjugate momentum is
\begin{align*}
\frac{\partial L}{\partial y'} = \frac{1}{\sqrt{-2gy}}\frac{y'}{\sqrt{1+y'^2}}
\end{align*}
and the Hamiltonian becomes
\begin{align*}
H = py' - L &= \frac{1}{\sqrt{-2gy}}\left(\frac{y'^2}{\sqrt{1+y'^2}} + \sqrt{1+y'^2} \right) \\
			&= \frac{1}{\sqrt{-2gy}}\frac{1}{\sqrt{1+y'^2}}\left(y'2 - 1 - y'^2 \right)
\end{align*}
\begin{align}
\to (-2gy)(1+y'^2) &= \frac{1}{H^2} \nonumber \\
(1 + y'^2)y &= -\frac{1}{2gH^2} \nonumber \\
(1 + y'^2)y &= -k^2 \label{eq:brachODE}.
\end{align}
$y(x)$ must satisfy a differential equation given by equation \ref{eq:brachODE}, where $k = \frac{1}{\sqrt{2g}H}$. 

\subsection{Parametric Solution}
The solution to the Brachistochrone probem can be written in parametric form as
\begin{align}
y &= \frac{1}{2}k^2(\theta-\sin\theta), \label{eq:brachsol1} \\
x &= \frac{1}{2}k^2(\cos\theta - 1), \label{eq:brachsol2}.
\end{align}
I am going to show that these indeed is a solution to \ref{eq:brachODE}.
\begin{equation*}
y' = \frac{dy}{dx} = \frac{dy}{d\theta}\frac{d\theta}{dx} = \frac{dy}{d\theta}/\frac{dx}{d\theta},
\end{equation*}
where
\begin{align*}
\frac{dx}{d\theta} &= \frac{1}{2}k^2(1-\cos\theta) = -y, \\
\frac{dy}{d\theta} &= -\frac{1}{2}k^2\sin\theta, \\
y' &= \frac{1}{2y}k^2\sin\theta
\end{align*}
This can now be inserted into the left hand side of the differential equation
\begin{align*}
(1 + y'^2)y &= (1+\frac{1}{4y^2}k^4\sin^2\theta)y \\ 
			&= \frac{1}{4}\frac{1}{y}(4y^2+k^4\sin^2\theta) \\
			&= \frac{1}{4}\frac{1}{y}(k^4(\cos\theta-1)^2 + k^4\sin^2\theta) \\
			&= \frac{1}{4}\frac{1}{y}k^4(\cos^2\theta - 2\cos\theta + 1 + \sin^2\theta) \\
			&= \frac{1}{y}k^2\frac{1}{2}k^2(1-\cos\theta) \\
			&= k^2\frac{1}{y}(-y) \\
			&= -k^2,
\end{align*}
and one can see that it equates to the right hand side.

It is worth checking how the  boundary conditions are taken care of in this situation,
\begin{align*}
x_a &= 0, \\
 \rightarrow \theta_a - \sin\theta_a = 0 \rightarrow \theta_a &= 0, \\
\rightarrow \cos\theta_a &= 1, \\
 \rightarrow y_a = \frac{1}{2}k^2(\cos\theta-1) &= 0.
\end{align*}
So point $A$ is fine. Now to look at point $B$ where the boundary condition is satisfied if the following equations are satisfied,
\begin{align*}
x_b = \frac{1}{2}k^2(\theta_b-\sin\theta_b), \\
y_b = \frac{1}{2}k^2(\cos\theta_b-1). \\
\end{align*}
This means that these two equations determine $k$ and $\theta_b$ and that $k$ is no longer an arbitrary constant.

\subsection{Cycloid}
The solution to the brachistrochrone problem, as given by the parameter equations \ref{eq:brachsol1} and \ref{eq:brachsol2} form a cycloid. This is the curve formed by a point of a rolling circle. I have already plotted such a curve in figure \ref{fig:brach}, but I have additionally made an animation in JavaScript which visualises the nature of a cycloid much better: \url{http://folk.uio.no/sebastwi/FYS3120/cycloid/}. A ``normal'' cycloid is a convex function formed by a circle rolling on the ground, while the solution of this problem is a concave cycloid formed by a circle rolling on the ceiling. 

\subsection{Optimal endpoint}
Assume the endpoint $B$ is the lowest point on the cycloid. This is the point where the rolling circle that constructs the cycloid has completed a half revolution, or $\theta_b = \pi$. The arc length is the path the circle has travelled given by $x_b = r\theta_b = r\pi$. The point that draws the cycloid must have moved from the top of the circle to the bottom, $y_b = -2r$. The two Cartesian coordinates for point $B$ (lowest point) must therefore have the following relation,
\begin{equation}
y_b = -\frac{2}{\pi}x_b
\end{equation}

The time it takes for the body to get to this point is
\begin{equation*}
T = \int_0^{x_b}\sqrt{\frac{1+y'^2}{-2gy}}dx = \int_0^{\pi}\sqrt{\frac{1+y'^2}{-2gy}}\frac{dx}{d\theta}d\theta
\end{equation*}
from previous calculations I know that $1-y'^2 = -k^2/y$ and that $dx/d\theta= -y$ 
\begin{equation}
\label{eq:cycloidperiod}
T = \int_0^\pi\sqrt{\frac{k^2}{2gy^2}}(-y)d\theta = \frac{k}{\sqrt{2g}}\int_0^\pi d\theta = \frac{\pi k}{\sqrt{2g}}.
\end{equation}
There appears to be a sign error here, but bear in mind that $y<0$, which means that $\sqrt{y^2}=-y$.

It would be nice to compare this with something, like a straight line. A straight line from $A$ to $B$ is given by
\begin{equation}
s = \sqrt{x_b^2+y_b^2}=\sqrt{\frac{1}{4}k^4\pi^2-k^4} = k^2\sqrt{\frac{\pi^2}{4}-1}.
\end{equation}
At constant acceleration $a$
\begin{equation*}
s = \frac{1}{2}aT'^2 \rightarrow T' = \sqrt{\frac{2s}{a}}
\end{equation*}
where $a = g\cos\alpha$, and $\alpha$ is the angle between $y$-axis and  the straight line, given by 
\begin{equation*}
\cos\alpha = \frac{\abs{y_b}}{s} = \frac{\abs{-k^2}}{k^2\sqrt{\frac{\pi^2}{4}-1}}=\frac{1}{\sqrt{\frac{\pi^2}{4}-1}}.
\end{equation*}
Putting everything together,
\begin{equation}
T' = \sqrt{\frac{2s}{a}} = \sqrt{\frac{2}{g}k^2\sqrt{\frac{\pi^2}{4}-1}\sqrt{\frac{\pi^2}{4}-1}} =k\sqrt{\frac{2}{g}\left(\frac{\pi^2}{4}-1 \right)}.
\end{equation}
In relation to the period for the cycloid path in equation \ref{eq:cycloidperiod} this is
\begin{equation}
\frac{T'}{T} = \frac{\frac{\pi k}{\sqrt{2g}}}{k\sqrt{k\frac{2}{g}\left(\frac{\pi^2}{4} +1\right)}} = \frac{\pi}{g}\left(\frac{\pi^2}{4} +1\right) \approx 1.11.
\end{equation}
In conclusion, the straight line path takes $1.11$ times longer time than the cycloid path.

\end{document}