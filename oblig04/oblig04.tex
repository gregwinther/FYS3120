\documentclass[11pt]{amsart}

\usepackage[utf8]{inputenc}
\usepackage{amsmath}
\usepackage{physics}
\usepackage{graphicx}

\renewcommand{\thesubsection}{\thesection.\alph{subsection}}

\title[Problem Sheet 4]{Hamiltonian Venture\\
	\hrulefill \small{ Problem Sheet 4: FYS3120 } \hrulefill}
	
\author[Winther-Larsen]{Sebastian G. Winther-Larsen}
\date{\today}

\begin{document}

\maketitle

\section{Constrained rod}

Figure \ref{fig:constrained_rod} shows a rod of lenth $b$ and evenly distributed mass $m$. One endpoint of the rod is constrained to move along a horizontal line, and the other endpoint is constrained to move along a vertical line. The two lines are in the same plane. There is no friction and the acceleration due to gravity is $g$.

\begin{figure}[ht]
\centering
	\includegraphics[width = 0.45\textwidth]{constrained_rod.png}
	\caption{Constrained rod}
	\label{fig:constrained_rod}
\end{figure}

\subsection{The Lagrangian and Lagrange's equation}
To fully describe the rod one needs one needs two translational coordinates and one rotational coordinate. The system has two constraints, so it is sufficient with one generalised coordinate, $\theta$, as the system only has one degree of freedom. The position of the left and right endpoint of the rod in terms of $\theta$ is
\begin{equation*}
(0, -b\cos \theta) \text{ and } (b\sin \theta, 0)
\end{equation*}
respectively. The position of the rods centre of mass must therefore be
\begin{equation}
\label{eq:com}
\vb{r} = (\frac{b}{2}\sin \theta, -\frac{b}{2}\cos \theta).
\end{equation}
It follows that 
\begin{equation*}
\dot{x} = \frac{b}{2}\dot{\theta}\cos \theta, \quad \dot{y} = \frac{b}{2}\dot{\theta}\sin \theta,
\end{equation*}
which gives
\begin{equation}
\label{eq:xandy}
\dot{x} + \dot{y} = \frac{b^2}{4}\dot{\theta}^2.
\end{equation}
The kinetic energy is the sum of translational and rotational energy\footnote{The moment of intertia for a rod rotating around its centre of mass is $I=\frac{mL^2}{12}$.}
\begin{equation}
\label{eq:travail1}
T 	= \frac{1}{2}m(\dot{x} + \dot{y}) + \frac{1}{2}I\dot{\theta}^2
	= \frac{1}{2}m\frac{b^2}{4}\dot{\theta}^2 + \frac{1}{2}\frac{mb^2}{12}
	= \frac{mb^2}{8}\dot{\theta}^2 + \frac{mb^2}{24}\dot{\theta}^2 
	= \frac{mb^2}{6}\dot{\theta}^2.
\end{equation}
The potential energy is
\begin{equation}
\label{eq:voltage1}
V 	= mgy = -\frac{1}{2}mg\cos \theta.
\end{equation}
The Lagrangian becomes
\begin{equation}
\label{eq:lagrangian1}
L = T - V = \frac{1}{6}mb^2\dot{\theta}^2 + \frac{1}{2}mgb \cos \theta 
\end{equation}

Lagrange's equation is given by
\begin{equation}
\label{eq:lagreqn}
\frac{d }{dt}\left(\frac{\partial L}{\partial \dot{\theta}} \right) - \frac{\partial L}{\partial \theta} = 0.
\end{equation}
Each part can be computed separately
\begin{align*}
\frac{\partial L}{\partial \theta} &= -\frac{1}{2} mgb \sin \theta \\
\frac{\partial L}{\partial \dot{\theta}} &= \frac{1}{3}mb^2\dot{\theta} \\
\frac{d }{dt}\left(\frac{\partial L}{\partial \dot{\theta}} \right) &= \frac{1}{3}mb^2 \ddot{\theta},
\end{align*}
and Lagrange's equation becomes
\begin{equation}
\frac{1}{3}mb^2\ddot{\theta} + \frac{1}{2}mgb \sin \theta = 0 \rightarrow 
\ddot{\theta} + \frac{3g}{2b}\sin \theta = 0
\end{equation}

\subsection{Equilibrium of the rod}
As usual, the system will tend towards a configuration where the potential, $V$, is as low as possible.  This point can be found by setting $\frac{\partial V}{\partial \theta} = 0$, but it is easy to see that it must be when $\theta = 0$.

When $\theta \to 0$, then $\sin \theta \to \theta$. Inserting this small-angle approximation into the Lagrange equation yields
\begin{equation}
\label{eq:lagapprox}
\ddot{\theta} + \frac{3g}{2b}\theta = 0,
\end{equation}
this equation corresponds to a harmonic oscillator with angular frequency $\omega = \sqrt{\frac{3g}{2b}}$. The period of an oscillation around the equilibrium orientation must then be
\begin{equation}
\label{eq:HOperiod}
T_0 = \frac{2\pi}{\omega} = 2\pi \sqrt{\frac{2b}{3g}}
\end{equation}

\subsection{Hamiltonian as constant of motion}
Since there $L$ has no explicit time dependence, there is a corresponding constant of motion. The Hamiltonian is related to the Lagrangian by the following equation
\begin{equation}
\label{eq:hamlang}
\frac{\partial H}{\partial t} = -\frac{\partial L}{\partial t} = 0.
\end{equation}
In this particular case the derivative of the Lagrangian with respect to time is equal to zero. It does not necessarily be so. The time derivative of the Hamiltonian must also be equal to zero, and is therefore a constant of motion. Now to find an expression for the Hamiltonian.
\begin{align}
H &= \sum_i \left( \frac{\partial L}{\partial \dot{q}_i} \dot{q}_i\right) - L \nonumber \\
 &= \frac{1}{3}mb^2\dot{\theta}^2 - \frac{1}{6}mb^2\dot{\theta}^2 - \frac{1}{2}mbg\cos\theta \nonumber \label{eq:hamiltonian} \\
 &= \frac{1}{6}mb^2\dot{\theta}^2 - \frac{1}{2}mbg\cos\theta \\
 &= T + V. \nonumber
\end{align}
One should take note that the Hamiltionian equals the total energy. The only force acting upon the system is gravity which is conservative. This implies that total energy is conserved, i.e. constant in time.

\subsection{Oscillation around equilibrium}
At the maximum angles $\theta_0$, the time derivative of the angle must be zero $\dot{\theta} = 0$. This reduces the Hamiltonian in this position to 
\begin{equation}
\label{eq:eqmHam}
H_0 = -\frac{1}{2}mbg \cos\theta_0.
\end{equation}
Because the energy i conserved (Hamiltonian is constant of motion) one can set expression \ref{eq:hamiltonian} equal to equation \ref{eq:eqmHam}.
\begin{align*}
H &= H_0 \\
\frac{1}{6}mb^2\dot{\theta}^2 - \frac{1}{2}mbg\cos\theta &= -\frac{1}{2}mbg\cos\theta_0 \\
\dot{\theta}^2 - \frac{3g}{2}\cos\theta &= - \frac{3g}{b}\cos\theta_0 \\
\frac{d\theta}{dt} = \dot{\theta} &= \pm \sqrt{\frac{3g}{b}}\sqrt{\cos\theta - \cos\theta_0} \\
\frac{dt}{d\theta} &= \pm \sqrt{\frac{b}{3g}} \frac{1}{\sqrt{\cos\theta - \cos\theta_0}}
\end{align*}
The last step will become apparent soon. From equation \ref{eq:HOperiod} one can set
\begin{equation}
\label{eq:H0period2}
\sqrt{\frac{b}{3g}} = T_0 \frac{1}{2\sqrt{2}\pi}
\end{equation}
which can be substituted into the expression above. The time it takes from the equilibrium position to the maximum angle is the sum of all infinitesimal angular displacements. Approximated by an integral this becomes
\begin{equation*}
T' = \int_0^{\theta_0}\frac{dt}{d\theta} d\theta = T_0 \frac{1}{2\sqrt{2}\pi}\int_0^{\theta_0}\frac{1}{\sqrt{\cos\theta - \cos\theta_0}}d\theta.
\end{equation*}
There are four such periods, $0 \to \theta_0$, $\theta_0 \to 0$, $0 \to -\theta_0$ and $-\theta_0 \to 0$. This means that the full period of the oscillatory motion is four times the integral above.
\begin{equation*}
T = 4T' = T_0 \frac{\sqrt{2}}{\pi} \int_0^{\theta_0}\frac{d\theta}{\sqrt{\cos\theta - \cos\theta_0}}.
\end{equation*}
The full space of ``swingable angles'' are included by setting $\theta_0 = \pi/2$
\begin{equation}
T = T_0\frac{\sqrt{2}}{\pi}\int_0^{\frac{\pi}{2}}\frac{d\theta}{\sqrt{\cos\theta}} = T_0\frac{\sqrt{2}}{\pi} \frac{\Gamma\left(\frac{1}{2} \right)\Gamma\left(\frac{1}{4} \right)}{2\Gamma\left(\frac{3}{4} \right)}
\end{equation}
the ratio $T/T_0$ becomes
\begin{equation}
\frac{T}{T_0} = \frac{1}{\sqrt{2}\pi}\frac{\Gamma\left(\frac{1}{2} \right)\Gamma\left(\frac{1}{4} \right)}{\Gamma\left(\frac{3}{4} \right)} \approx 1.18.
\end{equation}
I had to employ equation 103 on page 160 in K. Rottmann's collection of formulas and identities and a numerical solver to get here.

\section{Rotating pendulum}

\begin{figure}
\centering
	\includegraphics[width = 0.45\textwidth]{rotating_pendulum.png}
	\caption{Rotating pendulum}
	\label{fig:rotating_pendulum}
\end{figure}

\end{document}