\documentclass[11pt]{amsart}

\usepackage{physics}
\usepackage[utf8]{inputenc}
\usepackage{amsmath}

\renewcommand{\thesubsection}{\thesection.\alph{subsection}}

\title[FYS3120 Midterm]{Midterm Exam\\
		\hrulefill \large{ FYS3120 } \hrulefill}

\author[15137]{Candidate 15137}

\date{\today}

\begin{document}

\maketitle

\section{A (boring) Lagrangian}
A non-relativistic particle (no-potential) of mass $m$ is moving in three dimensions.

%% Sensible coordinate system and Lagrangian
\subsection{}

%% Conjugate momenta, compare with regular
\subsection{}

%% Cyclic coordinates
\subsection{}

%% Constants of motion
\subsection{}

%% Lagrangian for relativistic case. Demonstrate that it is invariant.
\subsection{}

%% Constant of motion for this Lagrangian
\subsection{}

%% Show that infinitesimal parameter must be anti-symmetric
\subsection{}
Consider a Lorentz transformation where the Lorentz transformation tensor is given as
\begin{equation}
\label{eq:lorentztransform}
L^\mu_{\ \nu} = \delta^{\mu}_{\ \nu} + \omega^{\mu}_{\ \nu}.
\end{equation}

Any particular Lorentz transformation must leave the line element $ds^2 = dx_{\mu}dx^{\mu}$ invariant,
\begin{align*}
g_{\mu\nu}dx'^{\mu}dx'^{\nu} = g_{\mu\nu} L^\mu_{\ \rho} L^\mu_{\ \sigma}dx^\rho dx^\sigma 
											&= g_{\rho\sigma}dx^\rho dx^\sigma \\
g_{\mu\nu}L^{\mu}_{\ \rho}L^{\nu}_{\sigma}	&= g_{\rho\sigma}
\end{align*}
Not to see if the Lorentz transformation in \ref{eq:lorentztransform} statisfies this requirement
\begin{align*}
g_{\mu\nu}  &= q_{\mu\nu}L^{\mu}_{\ \rho}L^{\nu}_{\ \sigma} \\
			&= g_{\mu\nu}(\delta^{\mu}_{\ \rho} + \omega^{\mu}_{\ \rho})(\delta^{\nu}_{\ \sigma} + \omega^{\nu}_{\ \sigma}) \\
			&= (\delta_{\nu\rho} + \omega_{\nu\rho})(\delta^{\nu}_{\ \sigma} + \omega^{\nu}_{\ \sigma}) \\
			&= \delta_{\nu\rho}\delta^{\nu}_{\ \sigma} + \delta_{\nu\rho}\omega^{\nu}_{\ \sigma} + \omega_{\nu\rho}\delta^{\nu}_{\ \sigma} + \omega_{\nu\rho}\omega^{\nu}_{\ \sigma} \\
			&= g_{\nu\rho}\delta^{\nu}_{\ \sigma} + g_{\nu\rho}\omega^{\nu}_{\ \sigma} + \omega_{\nu\rho}g^{\nu\gamma} g_{\gamma\sigma} + \omega^2_{\rho\sigma} \\
			& = \delta_{\rho\sigma} + \omega_{\rho\sigma} + \omega_{\sigma\rho} = g_{\rho\sigma} + g_{\nu\rho}(\omega^{\nu}_{\ \sigma} + \omega_{\sigma}^{\ \nu}),
\end{align*}
which only works if $\omega^{\mu}_{\nu}$ is antisymmetric, that is if $\omega^{\mu}_{\ \nu} = -\omega_{\nu}^{\ \mu}$.



\end{document}