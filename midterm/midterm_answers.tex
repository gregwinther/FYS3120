\documentclass[11pt]{amsart}

\usepackage{physics}
\usepackage[utf8]{inputenc}
\usepackage{amsmath}
\usepackage{tikz}
\usepackage{color}
\usepackage{cancel}

\definecolor{photon}{RGB}{0, 100, 0}
\definecolor{source}{RGB}{0, 200, 50}

\renewcommand{\thesubsection}{\thesection.\alph{subsection}}

\title[FYS3120 Midterm]{Midterm Exam\\	 
		\hrulefill \small{ FYS3120 } \hrulefill}

\author[15137]{Candidate 15137}

\date{\today}

\begin{document}

\maketitle

\section{A (boring) Lagrangian}
A non-relativistic particle (no-potential) of mass $m$ is moving in three dimensions.

%% Sensible coordinate system and Lagrangian
\subsection{}
A normal (boring), Cartesian coordinate system will do fine to study this problem in the first instance. One needs three coordinates to accurately describe the particle, and as there are no constraints on the particle, these three coordinates $x$, $y$ and $z$ are also the generalised coordinates.

The kinetic energy of the particle is given by
\begin{equation}
\label{eq:travail1}
T = \frac{1}{2}mv^2
\end{equation}
where $v = \abs{\vb{v}} = \sqrt{v_x^2 + v_y^2 + v_z^2} = \sqrt{\dot{x}^2 + \dot{y}^2 + \dot{z}^2}$, such that equation \ref{eq:travail1} becomes
\begin{equation}
T = \frac{1}{2}m(\dot{x}^2 + \dot{y}^2 + \dot{z}^2)
\end{equation}
there is no potential, so the Lagrangian is simply
\begin{equation}
L = T - V = T = \frac{1}{2}m(\dot{x}^2 + \dot{y}^2 + \dot{z}^2) = \frac{1}{2}mv^2
\end{equation}

%% Conjugate momenta, compare with regular
\subsection{}
The conjugate momenta are
\begin{align*}
p_x &= \frac{\partial L}{\partial \partial \dot{x}} = m\dot{x} \\
p_y &= \frac{\partial L}{\partial \partial \dot{y}} = m\dot{y} \\
p_z &= \frac{\partial L}{\partial \partial \dot{z}} = m\dot{z}, 
\end{align*}
or rather
\begin{equation}
p_v = \frac{\partial L}{\partial \vb{v}} = mv,
\end{equation}
which is exactly the same as the regular mechanical momentum.

%% Cyclic coordinates
\subsection{}
The position of the particle are cyclic coordinates, since
\begin{align*}
\frac{\partial L}{\partial x} = 0 \\
\frac{\partial L}{\partial y} = 0 \\
\frac{\partial L}{\partial z} = 0 
\end{align*}
alternatively
\begin{equation}
\frac{\partial L}{\partial \vb{r}} = \vb{0}.
\end{equation}

Only the position $\vb{r}$ varies with time, but the Lagrangian, interpreted as the physical situation, remains unchanged. This means that the initial value for the position does not determine the path of the particle. 

%% Constants of motion
\subsection{}
The Euler-Lagrange for this system equation is
\begin{equation}
\frac{\partial L}{\partial \vb{r}} - \frac{d}{dt}\frac{\partial}{\partial\dot{\vb{r}}} = 0.
\end{equation}
Since $\partial L / \partial \vb{r} = 0$, then
\begin{equation*}
\frac{d}{dt}\frac{\partial L}{\partial \dot{\vb{r}}} = 0,
\end{equation*}
and therefore the conjugate momentum $m\dot{\vb{r}}=m\vb{v}$ must be a constant of motion.

It now follows that the particle must move in a straight line such that
\begin{equation}
\vb{r} = \vb{r}_0 + \vb{v}t.
\end{equation}
My hunch is that the angular momentum of the particle must also be conserved,
\begin{align*}
\vb{L} = \vb{r} \times \vb{p} &= \vb{r}_0 \times \vb{p} \\
(\vb{r}_0 + \vb{v}t) \times \vb{p} &= \vb{r}_0 \times \vb{p} \\
\vb{r}_0 \times \vb{p} + \vb{v}t \times \vb{p} &= \vb{r}_0 \times \vb{p} \\
\vb{r}_0 \times \vb{p} &= \vb{r}_0 \times \vb{p}
\end{align*}
where $\vb{v}t \times \vb{p} = 0$ because $\vb{v}$ and $\vb{p}$ are parallel\footnote{This proof holds when the particle is viewed from any position \emph{not} on then path of the particle, such that $\vb{r} \not\parallel \vb{p}$.}. This means that the system is invariant under a rotation.

In conclusion, the conserved quantities are the momentum $\vb{p}$ and the angular momentum $\vb{L}$. Said in another way, this system can has both a translational and a rotational symmetry. Because of these two symmetric properties the system must have two corresponding quantities whose values are conserved in time\footnote{This last bit was an informal statement of Emmy Noether's theorem.}.

%% Lagrangian for relativistic case. Demonstrate that it is invariant.
\subsection{}
For any mechanical system there exist a certain integral $S$, called the action, which has minimum value for the actual motion, so that its variation is zero: $\delta S = 0$. To determine the action for a free particle, the integral must not depend on choice of reference system, because it must be invariant under Lorentz transformations. It follows that it must depend on a scalar. The action is
\begin{equation}
\label{eq:action1}
S = -k \int_a^b ds,
\end{equation}
where $k$ is a constant, $\int_a^b$ is the integral along the world line of the particles between two points $a$ and $b$, and $ds$ is a small displacement, and also a scalar of the right kind - being Lorentz invariant. The integral has a negative sign because this is the obvious way to ensure it has a minimum. 

The action integral can be represented as an integral with respect to time instead
\begin{equation}
S = \int_{t_1}^{t_2} Ldt,
\end{equation}
where $L$ is the Lagrangian of the mechanical system.

Now, a small detour. The the invariance of intervals gives
\begin{equation*}
ds^2 = c^2dt^2-dx^2-dy^2-dz^2=c^2dt'^2,
\end{equation*}
from  which
\begin{equation*}
dt' = dt\sqrt{1-\frac{dx^2+dy^2+dz^2}{c^2dt^2}}.
\end{equation*}
Furthermore,
\begin{equation*}
\frac{dx^2 + dy^2+dz^2}{dt^2} = v^2,
\end{equation*}
therefore
\begin{equation}
\label{eq:threetwo}
dt' = \frac{ds}{c} = dt\sqrt{1-\frac{v^2}{c^2}}
\end{equation}

Equation \ref{eq:threetwo} can be insterted into \ref{eq:action1} to give
\begin{equation}
S = - \int_{t_1}^{t_2} k c\sqrt{1-\frac{v^2}{c^2}}dt.
\end{equation}
Consequently, the Lagrangian of the free particle is $L = -kc\sqrt{1-v^2/c^2}$. One can expand $L$ in powers of $v/c$, ignoring higher order terms.
\begin{equation*}
L = -kc\sqrt{1-\frac{v^2}{c^2}} \approx -kc + \frac{kv^2}{2c}.
\end{equation*}
Constant terms in the Lagrangian do not affect the equations of motion and can be omitted. Compared with the classical expression $L = mv^2/2$, the constant must be $k=mc$.

The Lagrangian is
\begin{equation}
\label{eq:firstrelativisticlagrancanarian}
L = -mc^2\sqrt{1-\frac{v^2}{c^2}} = -\frac{mc^2}{\gamma}
\end{equation}

%% Constant of motion for this Lagrangian
\subsection{}
In the relativistic Lagrangian in equation \ref{eq:firstrelativisticlagrancanarian}, the position does not appear. Consequently, from the Lagrange-Euler equation
\begin{equation*}
\frac{\partial L}{\partial \vb{r}} = 0 \to \frac{d}{dt}\frac{\partial L}{\partial \vb{r}} = 0.
\end{equation*}
In other words, the conjugate momentum is a constant of motion. The conjugate momentum will take the expected form
\begin{equation}
\frac{\partial L}{\partial \vb{v}} = \frac{\partial}{\partial \vb{v}}\left(-mc^2\sqrt{1-\frac{\vb{v}^2}{c^2}} \right) = mc^2\frac{1}{\sqrt{1-\frac{\vb{v}^2}{c^2}}}\left(-\frac{v}{c^2}\right) = m\vb{v}\gamma,
\end{equation}
which is the usual way to write relativistic momentum (on non four-vector form).

%% Show that infinitesimal parameter must be anti-symmetric
\subsection{}
Consider a Lorentz transformation where the Lorentz transformation tensor is given as
\begin{equation}
\label{eq:lorentztransform}
L^\mu_{\ \nu} = \delta^{\mu}_{\ \nu} + \omega^{\mu}_{\ \nu}.
\end{equation}

Any particular Lorentz transformation must leave the line element $ds^2 = dx_{\mu}dx^{\mu}$ invariant,
\begin{align*}
g_{\mu\nu}dx'^{\mu}dx'^{\nu} = g_{\mu\nu} L^\mu_{\ \rho} L^\mu_{\ \sigma}dx^\rho dx^\sigma 
											&= g_{\rho\sigma}dx^\rho dx^\sigma \\
g_{\mu\nu}L^{\mu}_{\ \rho}L^{\nu}_{\sigma}	&= g_{\rho\sigma}
\end{align*}
To see if the Lorentz transformation in \ref{eq:lorentztransform} is invariant is must statisfy this requirement
\begin{align*}
g_{\rho\sigma}  &= q_{\mu\nu}L^{\mu}_{\ \rho}L^{\nu}_{\ \sigma} \\
			&= g_{\mu\nu}(\delta^{\mu}_{\ \rho} + \omega^{\mu}_{\ \rho})(\delta^{\nu}_{\ \sigma} + \omega^{\nu}_{\ \sigma}) \\
			&= (\delta_{\nu\rho} + \omega_{\nu\rho})(\delta^{\nu}_{\ \sigma} + \omega^{\nu}_{\ \sigma}) \\
			&= \delta_{\nu\rho}\delta^{\nu}_{\ \sigma} + \delta_{\nu\rho}\omega^{\nu}_{\ \sigma} + \omega_{\nu\rho}\delta^{\nu}_{\ \sigma} + \omega_{\nu\rho}\omega^{\nu}_{\ \sigma} \\
			&= g_{\nu\rho}\delta^{\nu}_{\ \sigma} + g_{\nu\rho}\omega^{\nu}_{\ \sigma} + \omega_{\nu\rho}g^{\nu\gamma} g_{\gamma\sigma} + \cancel{\omega^2_{\rho\sigma}} \\
			& = \delta_{\rho\sigma} + \omega_{\rho\sigma} + \omega_{\sigma\rho} = g_{\rho\sigma} + g_{\nu\rho}(\omega^{\nu}_{\ \sigma} + \omega_{\sigma}^{\ \nu}),
\end{align*}
which only works if $\omega^{\mu}_{\ \nu}$ is antisymmetric, that is if $\omega^{\mu}_{\ \nu} = -\omega_{\nu}^{\ \mu}$.

\subsection{}
A small Lorentz transformation between two reference frames changes the path $x^\mu(\tau)$ of a particle according to
\begin{equation}
\label{eq:pathperturbation}
\delta x^\mu(\tau) = x'^\mu(\tau) - x^\mu(\tau) = \omega^\mu_{\ \nu}x^\nu(\tau).
\end{equation}
This corresponds to a perturbation in the Lagrangian.

The variation of the Lagrangian is
\begin{equation*}
\delta L = \frac{\partial L}{\partial x^\mu} \delta x^\mu + \frac{\partial L}{\partial U^\mu} \delta U^\mu
\end{equation*}
inserting for $\delta x^\mu = \omega^\mu_{\ \nu} x^\nu$ from equation \ref{eq:pathperturbation} and 
\begin{equation*}
\delta U^\mu = \delta \frac{d x^\mu}{d t} = \frac{d}{d\tau}(\delta x^\mu) = \omega^\mu_{\ \nu}U^\nu, 
\end{equation*} 
which yields
\begin{equation}
\label{eq:lagrangeperturbartion}
\delta L = \left(\frac{\partial L}{\partial x^\mu}x^\nu + \frac{\partial L}{\partial U^\mu}U^\nu \right)x^\mu_{\ \nu}.
\end{equation}
This is the change in the Lagrangian as a consequence of the change in path.

\subsection{}
The Euler-Lagrange equations states
\begin{equation}
\label{eq:eulerlagrange1}
\frac{d}{d\tau}\left(\frac{\partial L}{\partial U^\mu} \right) = \frac{\partial L}{\partial x^\mu}.
\end{equation}
Inserting \ref{eq:eulerlagrange1} into \ref{eq:lagrangeperturbartion} gives
\begin{equation}
\delta L = \left(\frac{d}{d\tau}\left(\frac{\partial L}{\partial U^\mu}x^\nu \right) + \frac{\partial L}{\partial U^\mu}\frac{d}{d\tau}x^\nu \right)\omega^\mu_{\ \nu}
\end{equation}
using the product rule for derivation backwards gives
\begin{equation}
\delta L = \frac{d}{d\tau}\left(\frac{\partial L}{\partial U^\mu}x^\nu \right)\omega^\mu_{\ \nu} 
= \frac{1}{2}\frac{d}{d\tau}\left(\frac{\partial L}{\partial U^\mu}x^\nu + \frac{\partial L}{\partial U^\mu}x^\nu \right)\omega^\mu_{\ \nu}
\end{equation}
and finally ``letting everything run it's course''
\begin{align*}
\delta L &= \frac{1}{2}\frac{d}{d\tau}\left(\frac{\partial L}{\partial U^\mu}x^\nu + \frac{\partial L}{\partial U^\mu}x^\nu \right)\omega^\mu_{\ \nu} \\
		&= \frac{1}{2}\frac{d}{d\tau}\left(\frac{\partial L}{\partial U^\mu}x^\nu \omega^\mu_{\ \nu} -\frac{\partial L}{\partial U^\mu}x^\nu\omega_\nu^{\ \mu} \right) \\
		&= \frac{1}{2}\frac{d}{d\tau}\left(\frac{\partial L}{\partial g^{\rho\mu}U_\rho}x^\nu \omega^\mu_{\ \nu} -\frac{\partial L}{\partial g^{\rho\mu}U_\rho}x^\nu\omega_\nu^{\ \mu} \right) \\
		&= \frac{1}{2}\frac{d}{d\tau}\left(\frac{\partial L}{\partial U_\rho}x^\nu g_{\rho\mu} \omega^\mu_{\ \nu} -\frac{\partial L}{\partial U_\rho}x^\nu g_{\rho\mu}\omega_\nu^{\ \mu} \right) \\
		&= \frac{1}{2}\frac{d}{d\tau}\left(\frac{\partial L}{\partial U_\rho}x^\nu  \omega_{\rho\nu} -\frac{\partial L}{\partial U_\rho}x^\nu \omega_{\nu\rho} \right)		
\end{align*} 
changing indices back, writing $\mu$ instead of $\rho$, and moving $x^\nu$ to the left of the derivatives gives
\begin{equation*}
\delta L = \frac{1}{2}\frac{d}{d\tau}\left(x^\nu\frac{\partial L}{\partial U_\mu}  \omega_{\mu\nu} - x^\nu\frac{\delta L}{\delta U_\mu} \omega_{\nu\mu} \right).		
\end{equation*}
Switch indices of first term inside the parenthesis\footnote{This is okay because if one were to move $\partial U_\mu$ up from underneath the dividing line the index $\mu$ would change to an upstairs variant. This is the same as saying $\sum_i \sum_j x^i\frac{\partial L}{\partial U_j}\omega_{ji} = \sum_j \sum_i x^i\frac{\partial L}{\partial U_i}\omega_{ij}$}, and one ends up with an alternative expression for $\delta L$
\begin{equation}
\delta L = \frac{1}{2}\omega_{\nu\mu} \frac{d}{d\tau}\left(x^\mu\frac{\partial L}{\partial U_\nu} - x^\nu\frac{\partial L}{\partial U_\mu} \right)
\end{equation}

\subsection{}
For the path change to be invariant, there must, according to Hamilton's principle, be no change in the action 
\begin{equation}
\delta S = 0.
\end{equation}
This means that
\begin{equation}
\delta S = \int_{\tau_1}^{\tau_2}\delta L d\tau = \int_{\tau_1}^{\tau_2}\frac{1}{2}\omega_{\nu\mu} \frac{d}{d\tau}\left(x^\mu\frac{\partial L}{\partial U_\nu} - x^\nu\frac{\partial L}{\partial U_\mu} \right) d\tau = 0,
\end{equation}
which is true if
\begin{equation}
\delta L = \frac{1}{2}\omega_{\nu\mu} \frac{d}{d\tau}\left(x^\mu\frac{\partial L}{\partial U_\nu} - x^\nu\frac{\partial L}{\partial U_\mu} \right) = 0,
\end{equation}
or alternatively if
\begin{equation}
\label{eq:angulartensor1}
x^\mu\frac{\partial L}{\partial U_\nu} - x^\nu\frac{\partial L}{\partial U_\mu} = C
\end{equation}
where $C$ is a constant. Since $\frac{\partial L}{\partial U^\mu} = p^\mu$, then equation \ref{eq:angulartensor1} is the tensor form of the angular momentum
\begin{equation}
\ell^{\mu\nu} = x^\mu p^\nu - x^\nu p^\mu,
\end{equation}
leading one to conclude that the angular momentum is conserved because of the invariance under Lorentz transformation.


\section{Relativistics}
Two particles with mass $m$ and a photon is sent out from a source at the same time and in the positive $x$-direction in rest frame $S$ of the source. The massive particles are moving with constant velocity $v_1$ and $v_2>v_1$ in this frame. Figure \ref{fig:minkowski1} shows a Minkowski space-time diagram of the two particles, the photon and the source in the rest frame of the source $S$ and that of the slowest of the particles $S'$. 

\begin{figure}
\centering
\begin{tikzpicture}

	% Time lines
	\draw[-, dashed, color=red](-4.5,-4.5)--(4.5, 4.5);
	\draw[-, dashed, color=red](-4.5, 4.5)--(4.5, -4.5);
	
	% S' (Boost) coordinate system	
	\draw[->, color=gray] (-5, -1)--(5, 1) node[above]{$x'$};	
	\draw[->, color=gray] (-1, -5)--(1, 5) node[right]{$ct'$};

	% S	Coordinate system
	\draw[->, thick] (-5, 0)--(5, 0) node[right]{$x$};
	\draw[->, thick] (0, -5)--(0, 5) node[above]{$ct$};
	
	% Particle 1
	\draw[->, thick, color=blue] (0,0)--(0.6,3) node[left]{\small{1}};
	
	% Particle 2
	\draw[->, thick, color=blue] (0,0)--(1,3) node[right]{\small{2}};
	
	% Photon
	\draw[->, thick, color=photon] (0,0)--(3, 3) node[left]{\small{$\gamma$}};
	
	% Source
	\draw[->, thick, color=source] (0,0)--(0, 3) node[left]{\small{Source}};
	
\end{tikzpicture}
\caption{Minkowski space-time diagram of two massive particles (velocities $v_1$ and $v_2>v_1$) and a photon ($\gamma$) sent out from a source at origin in rest frame $S$. Rest frame $S'$ is that of particle 1.}
\label{fig:minkowski1}
\end{figure}

The relativistic formula for transition between two inertial frames is given by
\begin{equation}
x' = \gamma(x-vt), \quad t' = \gamma(t-\frac{v}{c^2}x),
\end{equation}
therefore, for an infinitesimal change in position coordinates we have
\begin{align*}
dx'  &= \gamma(dx - vdt) = \gamma(u-v)dt \\
dt'  &= \gamma(dt - \frac{v}{c^2}dx) = \gamma(1 - \frac{uv}{c^2})
\end{align*}
and from this follows that
\begin{equation*}
u' = \frac{dx'}{dt'} = \frac{u-v}{1-\frac{uv}{c^2}},
\end{equation*}
or specifically to this situation
\begin{equation}
\label{eq:boostvelocitychange}
v_2' = \frac{dx'}{dt'} = \frac{v_2-v_1}{1-\frac{v_2v_1}{c^2}}.
\end{equation}

The difference in rapidity of the two massive particles in the two different rest frames are
\begin{align}
S: &\quad \Delta\chi = \tanh^{-1}\left(\frac{v_2}{c}\right) - \tanh^{-1}\left(\frac{v_1}{c}\right) \\
S':&\quad \Delta\chi' = \tanh^{-1}\left(\frac{v'_2}{c}\right) - \tanh^{-1}\left(\frac{v'_1}{c}\right) = \tanh^{-1}\left(\frac{v'_2}{c}\right)
\end{align}
Rapidity differences should be unchanged by boosts no matter the reference frames, so
\begin{align*}
\tanh^{-1}\left(\frac{v_2}{c}\right) - \tanh^{-1}\left(\frac{v_1}{c}\right) &= \tanh^{-1}\left(\frac{v'_2}{c}\right) \\
\tanh^{-1}\left(\frac{\frac{v_2}{c}-\frac{v_1}{c}}{1-\frac{v_2v_1}{c^2}} \right) &= \tanh^{-1}\left(\frac{v'_2}{c}\right) \\
\tanh^{-1}\left(\frac{1}{c}\frac{v_2-v_1}{1-\frac{v_2v_1}{c^2}} \right) &= \tanh^{-1}\left(\frac{v'_2}{c}\right),
\end{align*}
inserting \ref{eq:boostvelocitychange} gives
\begin{align*}
\tanh^-1\left(\frac{v'_2}{c} \right) &= \tanh^-1\left(\frac{v'_2}{c} \right) \\
\chi &= \chi'.
\end{align*}
In conclusion, the rapidity difference is the same in the two rest frames $S$ and $S$.


\section{Finding the Shortest Way}
The shortest path between two points on a sphere. At some contstant radius $r$, some small movement in some direction on the sphere is
\begin{equation}
ds^2 = r^2d\theta^2 + r^2\sin^2\theta\phi^2 
\end{equation}
inserting for $d\phi = (d\phi/d\theta)d\theta = \dot{\phi}d\theta$ gives
\begin{equation}
ds = r\sqrt{1 +\sin^2\theta\dot{\phi}^2}d\theta
\end{equation}
A path is given by
\begin{equation}
S = \int ds = r \int_{\theta_A}^{\theta_B} \sqrt{1 + \sin^2\theta\dot{\phi}^2}d\theta
\end{equation}
where the integrand $F(\theta, \phi, \dot{\phi}) = \sqrt{1 + \sin^2\theta\dot{\phi}^2}$ does not depend explicitly on $\phi$. This implies that $\partial F/ \partial \dot{\phi}$ is constant, yielding
\begin{equation}
\frac{\partial F}{\partial \dot{\phi}} = \frac{2\sin^2\theta\dot{\phi}}{\sqrt{1 + \sin^2\theta\dot{\phi}^2}} = C' \rightarrow \frac{\sin^2\theta\dot{\phi}}{\sqrt{1 + \sin^2\theta\dot{\phi}^2}} = C  
\end{equation}
This can be rearranged
\begin{align*}
C^2 &= \frac{\sin^4\theta\dot{\phi}^2}{1 + \sin^2\theta\dot{\theta}^2} \\
C^2 + C\sin^2\theta\dot{\phi}^2 &= \sin^4\theta\dot{\phi}^2 \\
C^2 &= (\sin^4\theta - C\sin^2\theta)\dot{\phi}^2 \\
\dot{\phi}^2 &= \frac{C^2}{(\sin^4\theta - C\sin^2\theta)} \\
\dot{\phi} &= \frac{C}{\sin\theta\sqrt{\sin^2 - C}}
\end{align*}
Now this will be integrated from the starting point $(\theta_0, \phi_0) = (\pi/2,0)$ to the stop point $(\theta,\phi)$
\begin{align*}
\phi - \phi_0 &= \int_{\theta_0}^{\theta} \frac{r}{\sin\vartheta\sqrt{\sin^2\vartheta-r^2}}d\vartheta \\
\phi &= \int_{\frac{\pi}{2}}^{\theta} \frac{r}{\sin\vartheta\sqrt{\sin^2\vartheta-r^2}}d\vartheta.
\end{align*}
The integrand can be simplified by making the substitution $\sin^{-2}\vartheta = \csc^2\vartheta = 1 + \cot^2\vartheta$,
\begin{align*}
\frac{r}{\sin\vartheta\sqrt{\sin^2\vartheta-r^2}} &= \frac{r}{\sin^2\vartheta\sqrt{1 - \frac{r^2}{\sin^2\vartheta}}} \\
= \frac{\csc^2\vartheta r}{\sqrt{1-(1+\cot^2\vartheta)r^2}} &= \frac{\csc^2\vartheta}{\sqrt{1-r^2-r^2\cot^2\vartheta}}
\end{align*}
Then substitute for
\begin{align*}
u = \frac{r}{\sqrt{1-r^2}}\cot\vartheta &\leftrightarrow \cot\vartheta = \frac{u}{r}\sqrt{1-r^2} \\
\frac{du}{d\vartheta} = -\frac{r}{\sqrt{1-r^2}}\csc^2\vartheta &\leftrightarrow \csc^2\vartheta d\vartheta = -\frac{\sqrt{1-r^2}}{r}du,
\end{align*}
which gives
\begin{align*}
\phi &= \int_{\frac{\pi}{2}}^{\theta} \frac{rd\vartheta}{\sqrt{1-r^2-r^2\cot^2\vartheta}} \\
	&= \int_{u(\frac{\pi}{2})}^{u(\theta)} -\frac{\sqrt{1-r^2} r}{r\sqrt{1-r^2-r^2\cot^2\vartheta}} \\
	&= -\int_{u(\frac{\pi}{2})}^{u(\theta)} \frac{\sqrt{1-r^2}du}{\sqrt{1-r^2 - u^2(1-r^2)}} \\
	&= -\int_{u(\frac{\pi}{2})}^{u(\theta)} \frac{\sqrt{1-r^2} du}{\sqrt{1-r^2-u^2+u^2r^2}} \\
	&= -\int_{u(\frac{\pi}{2})}^{u(\theta)} \frac{\sqrt{1-r^2} du}{\sqrt{1-r^2}\sqrt{1-u^2}}\\
	&= -\int_{u(\frac{\pi}{2})}^{u(\theta)} \frac{du}{\sqrt{1-u^2}}	
\end{align*}
Karl Rottmann can tell me that $\frac{d}{dx}\arccos x = -(1-x^2)^{\frac{1}{2}}$ (p.130)
\begin{equation}
\phi(\theta) = \arccos(u(\theta)) - \arccos(u(\frac{\pi}{2})) = \arccos(\frac{r}{\sqrt{1-r^2}}\cot\theta)
\end{equation}


\end{document}