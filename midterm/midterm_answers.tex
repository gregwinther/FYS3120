\documentclass[11pt]{amsart}

\usepackage{physics}
\usepackage[utf8]{inputenc}
\usepackage{amsmath}

\renewcommand{\thesubsection}{\thesection.\alph{subsection}}

\title[FYS3120 Midterm]{Midterm Exam\\
		\hrulefill \large{ FYS3120 } \hrulefill}

\author[15137]{Candidate 15137}

\date{\today}

\begin{document}

\maketitle

\section{A (boring) Lagrangian}
A non-relativistic particle (no-potential) of mass $m$ is moving in three dimensions.

%% Sensible coordinate system and Lagrangian
\subsection{}

%% Conjugate momenta, compare with regular
\subsection{}

%% Cyclic coordinates
\subsection{}

%% Constants of motion
\subsection{}

%% Lagrangian for relativistic case. Demonstrate that it is invariant.
\subsection{}

%% Constant of motion for this Lagrangian
\subsection{}

%% Show that infinitesimal parameter must be anti-symmetric
\subsection{}
Consider a Lorentz transformation where the Lorentz transformation tensor is given as
\begin{equation}
\label{eq:lorentztransform}
L^\mu_{\ \nu} = \delta^{\mu}_{\ \nu} + \omega^{\mu}_{\ \nu}.
\end{equation}

Any particular Lorentz transformation must leave the line element $ds^2 = dx_{\mu}dx^{\mu}$ invariant,
\begin{align*}
g_{\mu\nu}dx'^{\mu}dx'^{\nu} = g_{\mu\nu} L^\mu_{\ \rho} L^\mu_{\ \sigma}dx^\rho dx^\sigma 
											&= g_{\rho\sigma}dx^\rho dx^\sigma \\
g_{\mu\nu}L^{\mu}_{\ \rho}L^{\nu}_{\sigma}	&= g_{\rho\sigma}
\end{align*}
To see if the Lorentz transformation in \ref{eq:lorentztransform} is invariant is must statisfy this requirement
\begin{align*}
g_{\rho\sigma}  &= q_{\mu\nu}L^{\mu}_{\ \rho}L^{\nu}_{\ \sigma} \\
			&= g_{\mu\nu}(\delta^{\mu}_{\ \rho} + \omega^{\mu}_{\ \rho})(\delta^{\nu}_{\ \sigma} + \omega^{\nu}_{\ \sigma}) \\
			&= (\delta_{\nu\rho} + \omega_{\nu\rho})(\delta^{\nu}_{\ \sigma} + \omega^{\nu}_{\ \sigma}) \\
			&= \delta_{\nu\rho}\delta^{\nu}_{\ \sigma} + \delta_{\nu\rho}\omega^{\nu}_{\ \sigma} + \omega_{\nu\rho}\delta^{\nu}_{\ \sigma} + \omega_{\nu\rho}\omega^{\nu}_{\ \sigma} \\
			&= g_{\nu\rho}\delta^{\nu}_{\ \sigma} + g_{\nu\rho}\omega^{\nu}_{\ \sigma} + \omega_{\nu\rho}g^{\nu\gamma} g_{\gamma\sigma} + \omega^2_{\rho\sigma} \\
			& = \delta_{\rho\sigma} + \omega_{\rho\sigma} + \omega_{\sigma\rho} = g_{\rho\sigma} + g_{\nu\rho}(\omega^{\nu}_{\ \sigma} + \omega_{\sigma}^{\ \nu}),
\end{align*}
which only works if $\omega^{\mu}_{\nu}$ is antisymmetric, that is if $\omega^{\mu}_{\ \nu} = -\omega_{\nu}^{\ \mu}$.

\subsection{}
A small Lorentz transformation between two reference frames changes the path $x^\mu(\tau)$ of a particle according to
\begin{equation}
\label{eq:pathperturbation}
\delta x^\mu(\tau) = x'^\mu(\tau) - x^\mu(\tau) = \omega^\mu_{\ \nu}x^\nu(\tau).
\end{equation}
This corresponds to a perturbation in the Lagrangian.

The variation of the Lagrangian is
\begin{equation*}
\delta L = \frac{\partial L}{\partial x^\mu} \delta x^\mu + \frac{\partial L}{\partial U^\mu} \delta U^\mu
\end{equation*}
inserting for $\delta x^\mu = \omega^\mu_{\ \nu} x^\nu$ from equation \ref{eq:pathperturbation} and 
\begin{equation*}
\delta U^\mu = \delta \frac{d x^\mu}{d t} = \frac{d}{d\tau}(\delta x^\mu) = \omega^\mu_{\ \nu}U^\nu 
\end{equation*} 
yields
\begin{equation}
\label{eq:lagrangeperturbartion}
\delta L = \left(\frac{\partial L}{\partial x^\mu}x^\nu + \frac{\partial L}{\partial U^\mu}U^\nu \right)x^\mu_{\ \nu},
\end{equation}
which is the change in the Lagrangian as a consequence of the change in path.

\subsection{}
The Euler-Lagrange equations states
\begin{equation}
\label{eq:eulerlagrange1}
\frac{d}{d\tau}\left(\frac{\partial L}{\partial U^\mu} \right) = \frac{\partial L}{\partial x^\mu}.
\end{equation}
Inserting \ref{eq:eulerlagrange1} into \ref{eq:lagrangeperturbartion} gives
\begin{equation}
\delta L = \left(\frac{d}{d\tau}\left(\frac{\partial L}{\partial U^\mu}x^\nu \right) + \frac{\partial L}{\partial U^\mu}\frac{d}{d\tau}x^\nu \right)\omega^\mu_{\ \nu}
\end{equation}
using the product rule for derivation backwards gives
\begin{equation}
\delta L = \frac{d}{d\tau}\left(\frac{\partial L}{\partial U^\mu}x^\nu \right)\omega^\mu_{\ \nu} 
= \frac{1}{2}\frac{d}{d\tau}\left(\frac{\partial L}{\partial U^\mu}x^\nu + \frac{\partial L}{\partial U^\mu}x^\nu \right)\omega^\mu_{\ \nu}
\end{equation}
and finally ``letting everything run it's course''
\begin{align*}
\delta L &= \frac{1}{2}\frac{d}{d\tau}\left(\frac{\partial L}{\partial U^\mu}x^\nu + \frac{\partial L}{\partial U^\mu}x^\nu \right)\omega^\mu_{\ \nu} \\
		&= \frac{1}{2}\frac{d}{d\tau}\left(\frac{\partial L}{\partial U^\mu}x^\nu \omega^\mu_{\ \nu} -\frac{\delta L}{\delta U^\mu}x^\nu\omega_\nu^{\ \mu} \right) \\
		&= \frac{1}{2}\frac{d}{d\tau}\left(\frac{\partial L}{\partial g^{\rho\mu}U_\rho}x^\nu \omega^\mu_{\ \nu} -\frac{\delta L}{\delta g^{\sigma\mu}U_\sigma}x^\nu\omega_\nu^{\ \mu} \right) \\
		&= \frac{1}{2}\frac{d}{d\tau}\left(\frac{\partial L}{\partial U_\rho}x^\nu g_{\rho\mu} \omega^\mu_{\ \nu} -\frac{\delta L}{\delta U_\sigma}x^\nu g_{\sigma\mu}\omega_\nu^{\ \mu} \right) \\
		&= \frac{1}{2}\frac{d}{d\tau}\left(\frac{\partial L}{\partial U_\rho}x^\nu  \omega_{\rho\nu} -\frac{\delta L}{\delta U_\sigma}x^\nu \omega_{\nu\sigma} \right)		
\end{align*} 
changing indices back, writing $\mu$ instead of $\rho,\sigma$, and moving $x^\nu$ to the left of derivatives gives
\begin{equation*}
\delta L = \frac{1}{2}\frac{d}{d\tau}\left(x^\nu\frac{\partial L}{\partial U_\mu}  \omega_{\mu\nu} - x^\nu\frac{\delta L}{\delta U_\mu} \omega_{\nu\mu} \right).		
\end{equation*}
Switch indices of first term inside the parenthesis\footnote{This is okay because if one were to move $\partial U_\mu$ up from underneath the dividing line the index $\mu$ would change to an upstairs variant. This is the same as saying $\sum_i \sum_j x^i\frac{\partial L}{\partial U_j}\omega_{ji} = \sum_j \sum_i x^i\frac{\partial L}{\partial U_i}\omega_{ij}$}, and one ends up with
\begin{equation}
\delta L = \frac{1}{2}\omega_{\nu\mu} \frac{d}{d\tau}\left(x^\mu\frac{\delta L}{\delta U_\nu} - x^\nu\frac{\partial L}{\partial U_\mu} \right)
\end{equation}

\section{Relativistics}

\section{Finding the Shortest Way}
The shortest path between two points on a sphere. At some contstant radius $r$, some small movement in some direction on the sphere is
\begin{equation}
ds^2 = r^2d\theta^2 + r^2\sin^2\theta\phi^2 
\end{equation}
inserting for $d\phi = (d\phi/d\theta)d\theta = \dot{\phi}d\theta$ gives
\begin{equation}
ds = r\sqrt{1 +\sin^2\theta\dot{\phi}^2}d\theta
\end{equation}
A path is given by
\begin{equation}
S = \int ds = r \int_{\theta_A}^{\theta_B} \sqrt{1 + \sin^2\theta\dot{\phi}^2}d\theta
\end{equation}
where the integrand $F(\theta, \phi, \dot{\phi}) = \sqrt{1 + \sin^2\theta\dot{\phi}^2}$ does not depend explicitly on $\phi$. This implies that $\partial F/ \partial \dot{\phi}$ is constant, yielding
\begin{equation}
\frac{\partial F}{\partial \dot{\phi}} = \frac{2\sin^2\theta\dot{\phi}}{\sqrt{1 + \sin^2\theta\dot{\phi}^2}} = C' \rightarrow \frac{\sin^2\theta\dot{\phi}}{\sqrt{1 + \sin^2\theta\dot{\phi}^2}} = C  
\end{equation}
This can be rearranged
\begin{align*}
C^2 &= \frac{\sin^4\theta\dot{\phi}^2}{1 + \sin^2\theta\dot{\theta}^2} \\
C^2 + C\sin^2\theta\dot{\phi}^2 &= \sin^4\theta\dot{\phi}^2 \\
C^2 &= (\sin^4\theta - C\sin^2\theta)\dot{\phi}^2 \\
\dot{\phi}^2 &= \frac{C^2}{(\sin^4\theta - C\sin^2\theta)} \\
\dot{\phi} &= \frac{C}{\sin\theta\sqrt{\sin^2 - C}}
\end{align*}

INTEGRATE!!

\end{document}