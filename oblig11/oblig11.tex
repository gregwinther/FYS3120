\documentclass[11pt]{amsart}

\usepackage[utf8]{inputenc}
\usepackage{amsmath}
\usepackage{physics}

\renewcommand{\thesubsection}{\thesection.\alph{subsection}}

\title[Electrodynamics]{Electrodynamics \\
	\hrulefill \small{ FYS3120: Problem Set 11 } \hrulefill}

\author[Winther-Larsen]{Sebastian G. Winther-Larsen}

\date{\today}

\begin{document}

\maketitle

\section{Simple Lagrangian Dynamics}

A non-relativistic particle, with electric charge $q$ and mass $m$ moves in a magnetic dipole field, given by the vector potential
\begin{equation}
\va{A} = \frac{\mu_0}{4\pi r^3}(\va{\mu} \cp \va{r}),
\end{equation}
where $\va{\mu}$ is the magnetic dipole moment of a static charge distribution centered at the origin.

\subsection{Lagrangian}
The Lagrangian is given by
\begin{equation}
\label{eq:lagrangian1}
L = T + q\va{v} \vdot \va{A}.
\end{equation}
The kinetic energy is simply $T = \frac{1}{2}m\va{v}^2$ while the potential is
\begin{align*}
q\va{v} \vdot \va{A} &= \frac{q\mu_0}{4\pi r^3}\va{v}\vdot(\va{\mu} \cp \va{r}) \\
					 &= \frac{q\mu_0}{4\pi r^3}\va{\mu}\vdot(\va{r}\cp\va{v}) \\
					 &= \frac{q\mu_0}{4\pi m r^3}\va{\mu}\vdot\va{\ell},
\end{align*}
using a vector triple product identity and $\va{\ell} = m\va{r}\cp\va{v}$. Inserting the parts into \ref{eq:lagrangian1} the Lagrangican becomes
\begin{equation}
\label{eq:lagranrian2}
L = \frac{1}{2}m\va{v}^2 + \frac{q\mu_0}{4\pi m r^3}\va{\mu}\vdot\va{\ell}.
\end{equation}

\subsection{Alternative Lagrangian}
We now make the assumption that the magnetic dipole moment is oriented along the $z$-axis and that the particle moves in the $(x,y)$-plane. In the following, $r = \abs{\va{r}}$ and  the angle $\phi$ between the $x$-axis and the position vector $va{r}$ are chosen as generalised coordinates.


\end{document}